\documentclass[conference]{IEEEtran}
\IEEEoverridecommandlockouts
% The preceding line is only needed to identify funding in the first footnote. If that is unneeded, please comment it out.
\usepackage{cite}
\usepackage{amsmath,amssymb,amsfonts}
\usepackage{algorithmic}
\usepackage{graphicx}
\usepackage{textcomp}
\usepackage{xcolor}
\usepackage[ngerman]{babel}
\usepackage[acronym]{glossaries}
%\usepackage{setspace}
\makeglossaries
\newacronym{sg}{SG}{Serious Games}
%BASH
\usepackage{listings}
\lstset{basicstyle=\ttfamily,
  showstringspaces=false,
  commentstyle=\color{red},
  keywordstyle=\color{blue}
}



\def\BibTeX{{\rm B\kern-.05em{\sc i\kern-.025em b}\kern-.08em
    T\kern-.1667em\lower.7ex\hbox{E}\kern-.125emX}}
\begin{document}

\title{Embedded System - Linux Tracing\\
%{\footnotesize \textsuperscript{*}Wissenschaftliches Arbeiten WS2021/2022 Fachhochschule Aachen}
%\thanks{Identify applicable funding agency here. If none, delete this.}
}


\author{\IEEEauthorblockN{Marcel Ochsendorf}
\IEEEauthorblockA{\textit{Embedded Systems} \\
\textit{Fachbereich 5 Elektrotechnik und Informationstechnik}\\ Fachhochschule Aachen, Deutschland}
\and
\IEEEauthorblockN{Muhammed Parlak}
\IEEEauthorblockA{\textit{Embedded Systems} \\
\textit{Fachbereich 5 Elektrotechnik und Informationstechnik}\\ Fachhochschule Aachen, Deutschland}
}

\maketitle

%-----------------------------------------------------------------------------------------------------------------------------------------------
% \gls{uscf}
\begin{abstract}

\end{abstract}

\begin{IEEEkeywords}
Linux Kernel Tracing, Embedded System, Embedded Programming
\end{IEEEkeywords}

\section{Einleitung}
Tracing ist die spezielle Verwendung der Protokollierung zur Aufzeichnung von Informationen über den Ausführungsablauf eines Programms.
Oft werden mit eigenständig hinzugefügte Print-Messages der Code debuggt. Somit verfolgt man die Anweisungen mit einem eigenem tracing-System.
Linux bringt einige eigenständige Tools mit, mit denen es möglich ist Vorgänge innerhalb von einem Embedded-System nachvollziehen und analysieren zu können.
Die Linux-Tracing Funktionalität und die bestehenden Tools, welche im Linux-Kernel integriert sind, helfen so dabei bei der Identifikation von Laufzeiten, Nebenläufigkeiten  und der Untersuchung von Latenzproblemen,
%-------------------------------------------------------------------------------------

\subsection{Tracing}


%ERLÄUTERUNGEN
Nachfolgend wird erleutert und an einem Beispiel demonstriert, wie das Linux-Tracing bei der Identifikation von Laufzeitproblemen eingesetzt werden kann.

%-----------------------------------------------------------------------------------------------------------------------------------------------
%METHODIKEN

\subsection{Ursprung}


\section{Grundlagen}

\subsection{Events}

\subsection{Kprobes}

Kprobes können dazu verwendet werden, Laufzeit und Performance-Daten des Kernels zu sammeln.
Der Vorteil and diesen ist, dass diese Daten ohne Unterbrechnung der Ausführung auf CPU-Instruktions-Ebene aggregiert werden können, anders wie bei dem Debuggen eines Programms mittels Breakpoints.

\subsubsection{CPU-Traps}


\begin{lstlisting}[language=bash,caption={bash version}]
    #!/bin/bash

\end{lstlisting}

\section{Tracing auf Mikrokontrollern}
%-----------------------------------------------------------------------------------------------------------------------------------------------
%SHADER

\section{Tools}

Allgemein sind keine speziellen Programme notwending um die Laufzeiteigenschaften eines Programms aufzuzeichnen.
Der Linux-Kernel bringt bereits alle nötigen Funktionalitäten mit. Jedoch gibt es Tools die eine visuelle Darstellung der aufgezeichneten Events ermöglichen.


\subsection{Trace-Log Aufzeichnung}

\subsubsection{ftrace}



\subsection{Visualisierung}

\subsubsection{Kernelshark}

\begin{figure}[htbp]
\centerline{\includegraphics[width=8cm]{kernelshark.png}}
\caption{Kernelshark}
\label{shadertool1_fig}
\end{figure}





\section{Interpretation des Kernel-Trace Ergebnisses}

%-----------------------------------------------------------------------------------------------------------------------------------------------
%VERGLEICHE

\section{Beispiel der Identifikation von Laufzeitproblemen}

\subsection{Ausgangsszenario}

Als Ausgangspunkt dieses Beispiels, soll das Laufzeitverhalten eines Programms auf einem Linux-System analysiert werden.
Die zugrunde liegende Software wurde bisher nur auf einem Linux-Realtime Kernel verwendet,
jedoch erfordert die Implementation neuer Features eine neuere Kernel-Version, welche noch nicht als RT-Version auf dem System zur Verfügung steht.
Somit soll ermittelt werden, ob die unmodifizierte Software eins zu eins auf dem neuen System lauffähig ist und die Laufzeitandorderungen erfüllt.


\subsection{Aufzeichnung mittels ftrace}

\subsection{Visualisierung und Beurteilung des Trace-Logs mittels kernelshark}
%-----------------------------------------------------------------------------------------------------------------------------------------------
%FAZIT

\section{Fazit}




%-----------------------------------------------------------------------------------------------------------------------------------------------
\begingroup

%\setstretch{1.05}
\begin{thebibliography}{00}
\bibitem{eduxrvrar2020usa} Thomas Alsop: "Leading applications of immersive technologies in the education sector in the next two years according to XR/AR/VR/MR industry
experts in the United States in 2020", in: Internetseite Statista, URL: https://www.statista.com/statistics/1185078/applications-immersive-technologies-xr-ar-vr-mr-education/, Abruf am 17.11.2021.

\vskip 0.05in
\bibitem{evallearningmixedreality2020} Y. M. Tang; K. M. Au; H. C. W. Lau; G. T. S. Ho, C. H. Wu; "Evaluating the effectiveness of learning design with mixed reality (MR) in higher education", Springer-Verlag London Ltd, 28.02.2020





\end{thebibliography}
\endgroup

\end{document}


