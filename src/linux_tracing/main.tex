\documentclass[conference]{IEEEtran}
\IEEEoverridecommandlockouts
% The preceding line is only needed to identify funding in the first footnote. If that is unneeded, please comment it out.
\usepackage{cite}
\usepackage{amsmath,amssymb,amsfonts}
\usepackage{algorithmic}
\usepackage{graphicx}
\usepackage{textcomp}
\usepackage{xcolor}
\usepackage[ngerman]{babel}
\usepackage[acronym]{glossaries}
%\usepackage{setspace}
\makeglossaries
\newacronym{sg}{SG}{Serious Games}



\def\BibTeX{{\rm B\kern-.05em{\sc i\kern-.025em b}\kern-.08em
    T\kern-.1667em\lower.7ex\hbox{E}\kern-.125emX}}
\begin{document}

\title{Embedded System - Linux Tracing\\
%{\footnotesize \textsuperscript{*}Wissenschaftliches Arbeiten WS2021/2022 Fachhochschule Aachen}
%\thanks{Identify applicable funding agency here. If none, delete this.}
}


\author{\IEEEauthorblockN{1\textsuperscript{st} Marcel Ochsendorf}
\IEEEauthorblockA{\textit{Fachhochschule Aachen} \\
\textit{Fachbereich 5 Elektrotechnik und Informationstechnik}\\
Aachen, Deutschland \\
marcel.ochsendorf@alumni.fh-aachen.de}
\and

}

\maketitle

%-----------------------------------------------------------------------------------------------------------------------------------------------
% \gls{uscf}
\begin{abstract}

\end{abstract}

\begin{IEEEkeywords}
Linux Kernel Tracing, Embedded System, Embedded Programming
\end{IEEEkeywords}

\section{Einleitung}
Tracing ist die spezielle Verwendung der Protokollierung zur Aufzeichnung von Informationen über den Ausführungsablauf eines Programms.
Oft werden mit eigenständig hinzugefügte Print-Messages der Code debuggt. Somit verfolgt man die Anweisungen mit einem eigenem tracing-System.
Linux bringt einige eigenständige Tools mit, mit denen es möglich ist Vorgänge innerhalb von einem Embedded-System nachvollziehen und analysieren zu können.

%-----------------------------------------------------------------------------------------------------------------------------------------------
%ERLÄUTERUNGEN
Nachfolgend wird erleutert und an einem Beispiel demonstriert, wie das Linux-Tracing bei der Identifikation von Laufzeitproblemen eingesetzt werden kann.

%-----------------------------------------------------------------------------------------------------------------------------------------------
%METHODIKEN

\subsection{Ursprung}



%-----------------------------------------------------------------------------------------------------------------------------------------------
%SHADER

\section{Tools}



\subsection{Trace-Log Aufzeichnung}


\subsection{Visualisierung}
\begin{figure}[htbp]
\centerline{\includegraphics[width=8cm]{kernelshark.png}}
\caption{Kernelshark}
\label{shadertool1_fig}
\end{figure}




%-----------------------------------------------------------------------------------------------------------------------------------------------
%VERGLEICHE

\section{Beispiel der Identifikation von Laufzeitproblemen}

\subsection{Ausgangsszenario}

\subsection{Aufzeichnung mittels ftrace}

\subsection{Visualisierung und Beurteilung des Trace-Logs}
%-----------------------------------------------------------------------------------------------------------------------------------------------
%FAZIT

\section{Fazit}




%-----------------------------------------------------------------------------------------------------------------------------------------------
\begingroup

%\setstretch{1.05}
\begin{thebibliography}{00}
\bibitem{eduxrvrar2020usa} Thomas Alsop: "Leading applications of immersive technologies in the education sector in the next two years according to XR/AR/VR/MR industry
experts in the United States in 2020", in: Internetseite Statista, URL: https://www.statista.com/statistics/1185078/applications-immersive-technologies-xr-ar-vr-mr-education/, Abruf am 17.11.2021.

\vskip 0.05in
\bibitem{evallearningmixedreality2020} Y. M. Tang; K. M. Au; H. C. W. Lau; G. T. S. Ho, C. H. Wu; "Evaluating the effectiveness of learning design with mixed reality (MR) in higher education", Springer-Verlag London Ltd, 28.02.2020





\end{thebibliography}
\endgroup

\end{document}


