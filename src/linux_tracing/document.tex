\hypertarget{einleitung}{%
\chapter{Einleitung}\label{einleitung}}

Tracing ist die spezielle Verwendung der Protokollierung zur
Aufzeichnung von Informationen über den Ausführungsablauf eines
Programms. Oft werden mit eigenständig hinzugefügten Print-Messages der
Code um Debug-Ausgaben erweitert. Somit verfolgt man die Anweisungen mit
einem eigenen Tracing-System.

Linux bietet einige eigenständige Tools, welche ermöglichen, Vorgänge
innerhalb von einem Linux-System nachzuvollziehen und analysieren zu
können. Die Linux-Tracing Funktionalität und die zur Verfügung stehenden
Tools helfen so bei der Identifikation von Laufzeiten, Nebenläufigkeiten
und der Untersuchung von Latenzproblemen. Hierzu sind bereits alle
nötigen Tools und Funktionalitäten im Linux-Kernel integriert.
\cite{dbgfs}

\hypertarget{relevanz}{%
\section{Relevanz}\label{relevanz}}

Bei Mikrokontrollern und auch im Zusammenhang mit
Echtzeit-Betriebssystemen ist jede Aktion, die Ausgeführt wird, von
hoher Bedeutung. Moderne Linux Systeme sind sehr komplex und bestehen
aus vielen Softwaremodulen, welche auf unterschiedlichsten Weisen
untereinander interagieren. Um diese Interaktionen nachzuvollziehen
können und um zu verstehen, wie sich Softwarekomponenten im Verbund
verhalten, ist es wichtig, das System systemnah zu debuggen, um diese
analysieren zu können. Oft können Fehler reproduziert und mit solchen
Analysen identifiziert werden. Zusätzlich besteht bei Custom Driver die
Möglichkeit während des Bootvorgangs zu debuggen.

\hypertarget{grundlagen}{%
\chapter{Grundlagen}\label{grundlagen}}

Um die Tracing-Funktionalität auf einem Linux-System verwenden zu
können, muss das System für deren Verwendung vorbereitet werden. Hierzu
muss unter anderem das Debug-Filesytem auf dem Ziel-System aktiviert
werden und die entsprechende Art des zu Tracings entsprechend der
Anwendung gewählt werden.

\hypertarget{ringbuffer}{%
\section{Ringbuffer}\label{ringbuffer}}

Bei einem Ringbuffer handelt sich um eine Datenstruktur, die das
asynchrone Lesen und Schreiben von Informationen erleichtert. Der Puffer
wird in der Regel als Array mit zwei Zeigern implementiert. Einem
Lesezeiger und einem Schreibzeiger. Man liest aus dem Puffer, indem man
den Inhalt des Lesezeigers liest und dann den Zeiger auf das nächste
Element erhöht und ebenso beim Schreiben in den Puffer mit dem
Schreibzeiger. So werden in der eingesetzten Ringbuffer-Implementierung
die Debug-Informationen gespeichert und ein Auslesen dieser ist mittels
der Einträge im Debug-\gls{fs} möglich.

\hypertarget{debug-filesystem}{%
\section{Debug-Filesystem}\label{debug-filesystem}}

Das Debug-\gls{fs} wurde in der Kernel-Version
\passthrough{\lstinline!2.6.10-rc3!}\cite{dbgfs} eingeführt. Es
bietet Zugriff auf Diagnose und Debug-Informationen auf Kernel-Ebene.

Ein Vorteil gegenüber des Prozess-\gls{fs}
\passthrough{\lstinline!/proc!} ist, dass jeder Entwickler hier auch
eigene Daten zur späteren Diagnose einpflegen kann. Um das Dateisystem
nutzen zu können, muss dies zuerst aktiviert werden. Nach der
Aktivierung stehen die Ordner unter dem angegebenen Pfad zur Verfügung.

\begin{lstlisting}[language=bash]
# ENABLE DEBUG FS
$ sudo mount -t debugfs debugfs /sys/kernel/debug
$ ls -lah /sys/kernel/debug | awk '{print $9}'
hid
usb 
Tracing
[...]
\end{lstlisting}

\hypertarget{tracing}{%
\section{Tracing}\label{tracing}}

Durch das Debug-\gls{fs} ist der Zugriff auf die Debug und insbesondere
auf die Tracing-Daten möglich. Im Debug-\gls{fs} ist nach der
Aktivierung die \passthrough{\lstinline!Tracing!}-Ordnerstruktur
vorhanden. In diesem werden verfügbaren Events in gruppiert in Ordnern
dargestellt, auf welche im späteren Verlauf reagiert werden kann. Zudem
können hier auch die Verfügbaren \passthrough{\lstinline!tracer!}
angezeigt und aktiviert werden, welche noch weitere Debugging-Optionen
bereitstellen.

\hypertarget{tracer}{%
\subsection{Tracer}\label{tracer}}

\begin{lstlisting}[language=bash]
# GET TRACERS
$ cat /sys/kernel/debug/tracing/available_tracers
hwlat blk mmiotrace function_graph wakeup_dl wakeup_rt wakeup function nop
# USE SPECIFIC TRACER
$ echo function_graph > /sys/kernel/debug/Tracing/current_tracer
# DISABLE TRACER USAGE
$ echo nop > /sys/kernel/debug/Tracing/current_tracer
\end{lstlisting}

\passthrough{\lstinline!tracer!} sind zusätzliche Tracing-Tools, welche
eine gezieltere Aggregierung von Events z.B. Filterung und somit
tiefergehende Analyse erlauben. Zum Beispiel erlaubt der
\passthrough{\lstinline!ftrace!}-Tracer eine detaillierte
Ereignis-Filterung auf spezifizierte
Events\cite{ftraceintroducation}.

Der \passthrough{\lstinline!function\_graph!}-Tracer gibt bei Verwendung
zusätzliche Informationen, wie z.B. die Laufzeit von einzelnen
Funktionen\cite{fgtrace}.

Auch kann dieser den Stacktrace und den Call-Stack übersichtlich
darstellen, indem hier die Namen der aufgerufenen Funktionen ausgegeben
werden.

\begin{lstlisting}[language=bash]
# CALL STACK USING FUNCTION_GRAPH TRACER
$ echo function_graph > /sys/kernel/debug/Tracing/current_tracer
$ cat /sys/kernel/debug/Tracing/trace
# tracer: function_graph
# CPU-  DURATION          FUNCTION CALLS
# |     |   |            |  |   |
 0)               |      getname() {
 0)   4.442 us    |         kmem_cache_alloc() {
 0)   1.382 us    |             __might_sleep();
 0)   2.478 us    |        }
 [...]
\end{lstlisting}

\hypertarget{events}{%
\subsection{Events}\label{events}}

Ein Event kann zum Beispiel durch das Lesen und Schreiben auf den System
\gls{i2c}-Bus vom Kernel ausgelöst werden. Wenn das Event im
Debug-\gls{fs} aktiviert wurde, stellt dieses die Informationen des
Events bereit. Je nach Typ können unterschiedlichste Informationen dem
Nutzer bereitgestellt werden. In der
\passthrough{\lstinline!ls!}-Ausgabe des
\passthrough{\lstinline!events!}-Ordners des Debug-\gls{fs} ist zu
sehen, welche Events abgefangen und mittels der Linux-Tracing-Tools
protokoliert werden können.

\begin{lstlisting}[language=bash]
# GET AVAILABLE EVENT LIST
$ cd /sys/kernel/debug/tracing/events
$ ls -lah | awk '{print $9}'
alarmtimer
drm
exceptions
ext4 #READPAGE, WRITEPAGE, ERROR, FREE_BLOCKS
filelock
filemap
gpio #GPIO_DIRECTION, GPIO_VALUE
hda
i2c
irq
net
smbus #READ, WRITE, REPLY
sock #STATE_CHANGED, EXCEED_BUFFER_LIMIT, REC_QUEUE_FULL
spi
tcp
timer #TIMER_STOP, TIMER_INIT, TIMER_EXPIRED
[...]
\end{lstlisting}

Alle Events sind in Gruppen gebündelt. Alle Events, welche das
\passthrough{\lstinline!ext4!}-\gls{fs} betreffen, befinden sich im
\passthrough{\lstinline!ext4!}-Ordner. Die Auflistung zeigt einige der
für das \passthrough{\lstinline!ext4!} zur Verfügung stehenden Events.
Zudem befinden sich zwei zusätzliche Dateien
\passthrough{\lstinline!enable!}, \passthrough{\lstinline!filter!}in
diesem Ordner. Durch diese ist es später möglich anzugeben, ob dieses
Event aufgezeichnet werden soll.

\begin{lstlisting}[language=bash]
$ cd /sys/kernel/debug/tracing/events/ext4
$ ls -lah | awk '{print $9}'
# EVENTS FOR EXT4
ext4_write_end
ext4_writepage
ext4_readpage
ext4_error
[...]
# INTERFACE FOR EVENT SETUP
enable
filter
format
\end{lstlisting}

Die optionale \passthrough{\lstinline!format!}-Datei kann zusätzliche
Informationen durch das Event bereitgestellte Format der Ausgabe
wiedergeben. Das folgende Beispiel zeigt das Ausgabeformat für das
Scheduler-Wakeup \passthrough{\lstinline!sched\_wakeup!}-Event. Somit
kann nicht nur in Erfahrung gebracht werden, wann und ob das Event
ausgelöst hat, sondern es können auch weitere Event-Spezifische
Informationen durch das Event gemeldet werden.

\begin{lstlisting}[language=bash]
$  cat /sys/kernel/debug/tracing/events/sched/sched_wakeup/format
ID: 318
format:
    field:unsigned short common_type;   offset:0;   size:2; signed:0;
    field:unsigned char common_flags;   offset:2;   size:1; signed:0;
    field:unsigned char common_preempt_count;   offset:3;   size:1; signed:0;
    field:int common_pid;   offset:4;   size:4; signed:1;
    field:char comm[16];    offset:8;   size:16;    signed:1;
    field:pid_t pid;    offset:24;  size:4; signed:1;
    field:int prio; offset:28;  size:4; signed:1;
    field:int success;  offset:32;  size:4; signed:1;
    field:int target_cpu;   offset:36;  size:4; signed:1;
\end{lstlisting}

\hypertarget{abfangen-von-events}{%
\section{Abfangen von Events}\label{abfangen-von-events}}

Um ein \passthrough{\lstinline!event!}\cite{events} abfangen zu
können, muss dies zuerst für die gewünschten Events aktiviert werden.
Hierzu werden die Event-Interface-Dateien verwendet, welche sich in
jeder Event-Gruppe befinden. Die simpelste Methode ist es, eine
\passthrough{\lstinline!1!} oder \passthrough{\lstinline!0!} in die
\passthrough{\lstinline!enable!}-Datei der Gruppe zu schreiben. Ein
spezifisches Event kann mit der gleichen Methode aktiviert werden.
Hierzu wird die \passthrough{\lstinline!enable!}-Datei im eigentlichen
Event-Ordner verwendet anstatt jene, welche sich in der Event-Gruppe
befindet.

\begin{lstlisting}[language=bash]
$ cd /sys/kernel/debug/tracing/events/ext4
# ENABLE ALL EVENTS FROM THIS GROUP
$ echo 1 > ./enable
# DISBALE ALL EVENTS
$ echo 0 > ./enable
# ENBABLE SPECIFIC EVENT
$ echo 1 > ./ext4_readpage/enable
$ echo 1 > ./ext4_writepage/enable
\end{lstlisting}

Nach dem Aktivieren der Events, können diese z.B.
\passthrough{\lstinline!Trace-Log!} Aufgezeichnet oder anderweitig
ausgegeben werden.

\hypertarget{probes}{%
\section{Probes}\label{probes}}

\hypertarget{kernel-probes}{%
\subsection{Kernel-Probes}\label{kernel-probes}}

\passthrough{\lstinline!kprobes!}\cite{kprobes} können dazu
verwendet werden, Laufzeit und Performance-Daten des Kernels zu sammeln.
Der Vorteil dieser ist, dass Daten ohne Unterbrechung der Ausführung auf
CPU-Instruktions-Ebene aggregiert werden können, anders als bei dem
Debuggen eines Programms mittels Breakpoints. Ein weiterer Vorteil ist,
dass das Registrieren der Kprobes dynamisch zur Laufzeit und ohne
Änderungen des Programmcodes geschieht. Somit ist es möglich, zu
verschiedenen Laufzeiten des zu analysierenden Systems oder Programms,
Daten zu verschiedenen Laufzeiten gezielt sammeln zu können.

Der \passthrough{\lstinline!kretprobes!}\cite{kretprobes} ermöglicht
uns auf den Rückgabewert jeder Kernel- oder Modulfunktion zuzugreifen!
Die Möglichkeit, den Rückgabewert einer bestimmten Funktion dynamisch
nachzuschlagen, kann in einem Debug-Szenario ein entscheidender Vorteil
sein.

\hypertarget{user-level-probes}{%
\subsection{User-Level-Probes}\label{user-level-probes}}

Eine Weiterentwicklung zu den \passthrough{\lstinline!kprobes!} sind die
\passthrough{\lstinline!uprobes!}. Mit diesem können zur Laufzeit Events
in eine Applikation eingebunden werden. Wenn ein
\passthrough{\lstinline!uprobes!} hinzugefügt werden soll, muss davor
noch was gemacht werden.

\begin{lstlisting}[language={C++}]
//test.c
#include <stdio.h>
int main(void)
{
    int i;
    for (i = 0; i < 5; i++)
        printf("Hello uprobe\n");
    return 0;
}
\end{lstlisting}

Bei der Nutzung von \passthrough{\lstinline!kprobes!} kann ein einfacher
Symbolnamen spezifiziert werden. Aufgrund der Tastsache, dass alle
Applikationen ihren eigenen virtuellen Adressraum besitzen, haben diese
auch einen anderen Adressbasis. Beim Erzeugen eines
\passthrough{\lstinline!uprobes!} wird das Adressoffset im Textsegment
der jeweiligen Applikation benötigt. Der obere C++-Code, stellt ein
einfaches Beispiel dar, indem die
\passthrough{\lstinline!printf!}-Anweisung, mittels einer
\passthrough{\lstinline!uprobe!} aufgezeichnet werden soll. Der
Adressoffset kann mittels \passthrough{\lstinline!objdump!} und dem Pfad
des zu analysierenden Programms. Danach kann die
\passthrough{\lstinline!uprobe!} im Debuf-\gls{fs} registriert werden
unter Angabe des Offsets. Als letzter Schritt, muss das neu erstellte
\passthrough{\lstinline!uprobe!}-Event noch aktiviert werden und die
Aufzeichnung oder Ausgabe der \passthrough{\lstinline!uprobe!}.

\begin{lstlisting}[language=bash]
# CREATE EXECUTE OBJECT
$ gcc ./test.c -o ./tmp/test
# GET OFFSET
$ objdump -F -S -D ./test | less | grep main
0000000000001149 <main> (File Offset: 0x1149):
# REGISTER A uprobe_event
$ echo "p:my_uprobe /tmp/test:0x1149" > /sys/kernel/debug/tracing/uprobe_events
# ACTIVATE UPROBE EVENTS
$ echo 1 > /sys/kernel/tracing/events/uprobes/enable
# EXECUTE PROGRAM
$ /root/hello
Hello uprobe
[...]
# PRINT TRACED EVENTS
$ cat /sys/kernel/debug/tracing/trace
# tracer: nop
# TASK-PID  CPU#  TIMESTAMP  FUNCTION
#  |   |     |        |         |
test-24842 [006] 258544.995456: printf: [...]
[...]
\end{lstlisting}

Ein weiterer Anwendungsfall ist die Inspektion von System-Bibliotheken.

\hypertarget{ressourcen}{%
\section{Ressourcen}\label{ressourcen}}

Beim Tracing werden zusätzliche Ressourcen benötigt, die Auswirkungen
auf die reale Ausführzeit haben. Bei Echtzeitbetriebsystemen können
diese zu Problemen führen, wenn dieses bereits mit den maximalen
Ressourcen arbeitet.

Die Aufzeichnung eines Trace-Logs benötigt je nach aktivierten Events,
eine nicht unerhebliche Menge an Speicherplatz auf dem System.
Zusätzlich muss das Medium auf welchem die Logs gespeichert werden
sollen, ein Mindestmaß an Bandbreite zur verfügung stellen.

Wird zusätzlich die Auswertung auf dem zu analysierenden Gerät
durchgeführt, benötigt diese weitere Ressourcen und kann somit den
Betrieb und die Aufzeichnung beeinflussen.

Um diese nachteiligen Effekte zu minimieren, sollte die Auswertung auf
einem seperaten Gerät durchgeführt werden. Vor der Aufzeichnung sollen
nur die Events aktiviert werden, welche im Fokus der analyse stehen.
Somit kann Bandbreite und Speicherplatz verringert werden, welche die zu
analysierende Anwendung ggf. selbst benötigt.

\hypertarget{tools}{%
\chapter{Tools}\label{tools}}

Allgemein sind keine speziellen Programme notwendig, um die
Laufzeiteigenschaften eines Programms aufzuzeichnen. Der Linux-Kernel
bringt bereits alle nötigen Funktionalitäten mit, jedoch gibt es Tools,
welche eine visuelle Darstellung der aufgezeichneten Events ermöglichen.
Somit kann die Aufzeichnung headless auf dem Ziel-System geschehen und
die spätere Analyse mit entsprechenden Tools auf einem anderen System
erfolgen.

\hypertarget{trace-log-aufzeichnung}{%
\section{Trace-Log Aufzeichnung}\label{trace-log-aufzeichnung}}

\begin{figure}
\centering
\includegraphics{images/trace-log-print.png}
\caption{Trace-Log \label{trace-log}}
\end{figure}

Für die Log-Aufzeichnung wird der zuvor beschriebene Ringbuffer genutzt.
Das Aufzeichnen in den Ringpuffer ist standardmäßig aktiviert. Kann aber
bei Bedarf deaktiviert werden.

\begin{lstlisting}[language=bash]
$ echo 1 > Tracing on
$ echo 0 > Tracing on
\end{lstlisting}

Mit dem folgenden Befehl kann der Inhalt des Ringbuffers auch während
einer Aufzeichnung, ausgebeben werden. Somit sind im Allgemeinen keine
besonderen Tools notwendig. Anwendungen zum Ausgeben von Dateien wie
z.B.\passthrough{\lstinline!cat!} oder \passthrough{\lstinline!less!},
welche sich auch auf kleinen Systemen befinden, sind
ausreichend.\ref{trace-log}

\begin{lstlisting}[language=bash]
$ less /sys/kernel/Tracing/trace
\end{lstlisting}

Das Lesen während einer Aufzeichnung mittels Trace hat keinerlei
Einfluss auf den Inhalt des Ringbuffers.

Die bisherigen Aufzeichnungen der Ereignisse können mit dem Leeren der
\passthrough{\lstinline!trace!}-Datei entfernt werden:

\begin{lstlisting}[language=bash]
$ echo ''> trace
\end{lstlisting}

Um einen Überlauf an Informationen zu verhindern kann die Aufzeichnung
auch konsumierend gelesen werden. Somit werden beim Lesen zeitgleich
diese aus dem Ringbuffer entfernt.

Ein weiterer Kernpunkt ist, dass in Mehrkernsystemen für jeden einzelnen
Core ein separater Ringbuffer existiert. Damit die Analyse von
verschiedenen Events getrennt werden kann, kann mit jeder weiteren
Instanz pro Core ein weiterer Ringbuffer angelegt werden. Dies erfolgt
im Untervereziechnis \passthrough{\lstinline!instances/!}. Das
Debug-\gls{fs} legt nach dem Anlegen des Ordners, die benötigten Dateien
wie die \passthrough{\lstinline!trace!}-Datei automatisch an. Alle
weiteren Operationen können dann auch auf dieser ausgeführt werden.

\begin{lstlisting}[language=bash]
$ cd  /sys/kernel/tracing/instances
$ mkdir ./inst0
$ mkdir ./inst1
\end{lstlisting}

\hypertarget{trace-cmd}{%
\subsection{trace-cmd}\label{trace-cmd}}

\begin{figure}
\centering
\includegraphics{images/trace-cmd.png}
\caption{trace-cmd Report \label{trace-cmd-report}}
\end{figure}

Das Tool \passthrough{\lstinline!trace-cmd!}\cite{trace-cmd} ist das
bekannteste und meistgenutzte Hilfsmittel zur Aufzeichnung. Dies ist ein
Kommandozeilenwerkzeug, das auf den meisten gängigen
Linux-Distributionen bereits vorinstalliert ist.

Mit dem letzten Befehl werden alle Events vom Scheduler aufgezeichnet.
Dabei werden während der Aufzeichnung kontinuierlich die Ringbuffer in
konsumierender Form ausgelesen und in die Datei
\passthrough{\lstinline!trace.dat!} geschrieben, falls mit dem
\passthrough{\lstinline!-o!} keine eigene Datei eingegeben wurde. Als
Informationen werden zu dem Inhalt des Ringbuffers auch zusätzlich
notwendige Informationen über das Target, für die Auswertung auf
beliebigen Systemen gespeichert.

\begin{lstlisting}[language=bash]
# CHECK IF TRACING IS ENABLED
$ sudo mount | grep tracefs
none on /sys/kernel/tracing type tracefs (rw,relatime,seclabel)
## ONLY SCHEDULER EVENTS
$ echo sched_wakeup >> /sys/kernel/debug/tracing/set_event
## ALL EVENTS USING set_event
$ echo *:* > /sys/kernel/debug/tracing/set_event
# RECORD
$ trace-cmd record ./program_executable
# RECORD SPECIFIC EVENT
$ trace-cmd record -e sched ./program_executable
# USING A ADDITIONAL TRACER
$ trace-cmd -t function ./program_executable
\end{lstlisting}

Die \passthrough{\lstinline!trace-cmd!} Konsolenanwendung dient nicht
nur zur Aufzeichnung der Trace-Events, sondern bietet auch die
Möglichkeit aufgezeichnete Reports visuell darzustellen. Die Ausgabe
erfolgt mit dem Befehl
\passthrough{\lstinline!trace-cmd report [-i <Dateiname>]!} als Tabelle
in der Konsole und ist somit rein Textbasiert\ref{trace-cmd-report}. Auf
Aufzeichnungen können zusätzliche Filter angewendet werden, um die Suche
auf bestimmten Ereignissen einzugrenzen. Mit dem Tool ist es einfach die
Teilschritte zu automatisieren.

\hypertarget{bpftrace}{%
\subsection{bpftrace}\label{bpftrace}}

Seit der Kernelversion \passthrough{\lstinline!>4.x!}, kann ein weiteres
Tool mit dem Namen \passthrough{\lstinline!bpftrace!}\cite{bpftrace}
verwendet werden. Dieses bietet jedoch zusätzlich eine eigene
Skriptsprache, mit welcher nicht nur Aggregation, sondern auch die
Eventfilter und die Verarbeitung der Ergebnisse automatisiert werden
können.

\begin{lstlisting}[language=bash]
# Block I/O latency as a histogram EXAMPLE
$ wget https://raw.githubusercontent.com/iovisor/bpftrace/master/tools/biolatency.bt
$ bftrace ./biolatency.bt
@usecs:
[512, 1K)             10 |@                       |
[ 1K, 2K)            426 |@@@@@@@@@@@@@@@@@@      |
[2K, 4K)             230 |@@@@@@@@@@@@@@          |
[4K, 8K)               9 |@                       |
[8K, 16K)            128 |@@@@@@@@@@@@@@@         |
[16K, 32K)            68 |@@@@@@@@                |
[...]
\end{lstlisting}

\hypertarget{kernelshark}{%
\subsection{Kernelshark}\label{kernelshark}}

Das zuvor vorgestellte \passthrough{\lstinline!tace-cmd!} ist wie oben
erwähnt nur ein textbasiertes Analysetool. Kernelshark Tool bietet dem
Anwender die Möglichkeit, die Trace-Aufzeichnungen grafisch zu
analysieren. Dabei sind die beiden Tools aufeinander abgestimmt und
werden gemeinsam entwickelt. Auch dieses Tool ist in den meisten Linux
Distributionen vorinstalliert.

Das vom trace-cmd erzeugte \passthrough{\lstinline!trace.dat-Format!}
wird im Kernelshark als Eingabe erwartet. Wenn im folgendem ersten
Befehl nichts eingegeben, dann wird nach der entsprechenden
\passthrough{\lstinline!trace.dat!} im Verzeichnis gesucht.

\begin{lstlisting}[language=bash]
# OPEN KERNELSHARK WITH trace.dat
$ kernelshark
# OPEN KERNELSHAR WITH SPECIFIED TRACELOG
$ kernelshark -i <Dateiname>
\end{lstlisting}

Im Folgenden ist die grafische Darstellung\ref{kernelshark} zu sehen.
Dabei besitzt jeder Task einen eigenen Farbton. Für jede CPU wird eine
eigene Zeile dargestellt.

\begin{figure}
\centering
\includegraphics{images/kernelshark.png}
\caption{Kernelshark \label{kernelshark}}
\end{figure}

\hypertarget{beispiel---tcp-paketanalyse}{%
\chapter{Beispiel - TCP
Paketanalyse}\label{beispiel---tcp-paketanalyse}}

Dieses Beispiel zeigt, wie der Empfang von \gls{tcp}-Netzwerkpaketen auf
Paketverlust auf einem System überprüft werden kann. Hierbei soll
analysiert werden, wie das System auf eine unerwartet große Menge an
\gls{tcp}-Paketen reagiert. Dies kann zum Beispiel bei
\gls{iot}-Anwendungen der Fall sein, bei denen das \gls{mqtt}-Protokoll
verwendet wird. Hierbei können viele kleine Netzwerkpakete von
\gls{iot}-Sensoren einen starken Traffics am Server zur Folge haben.

\hypertarget{bpftrace-installation}{%
\section{bpftrace Installation}\label{bpftrace-installation}}

Dabei wird auf dem zu analysierenden System
\passthrough{\lstinline!bpftrace!}\cite{bpftrace} verwendet. Unter
Debian-Systemen kann dies einfach über den APT-Package-Manager
installiert werden, jedoch ist diese Version, welche in der Registry
hinterlegt ist, meist nicht aktuell. Das folgende Beispiel erfordert die
Version \passthrough{\lstinline!>= 0.14!}. Somit muss
\passthrough{\lstinline!bpftrace!} aus den Quellen gebaut werden, da in
der APT-Registry nur die Version \passthrough{\lstinline!\~0.11!} zur
Verfügung stand.

\begin{lstlisting}[language=bash]
# INSTALL FROM SOURCE
$ git clone https://github.com/iovisor/bpftrace ./bpftrace
$ cd ./bpftrace && mkdir -p build
$ cmake -DCMAKE_BUILD_TYPE=Release . && make -j20
$ sudo make install
# GET TCP DROP EXAMPLE
$ cp ./bpftrace/tools/tcpdrop.bt ~/
\end{lstlisting}

\hypertarget{tcpdrop.bt}{%
\section{TCPDROP.BT}\label{tcpdrop.bt}}

Das \passthrough{\lstinline!tcpdrop.bt!} Skript, welches in diesem
Beispiel verwendet wird, registriert eine
\passthrough{\lstinline!kprobe!} auf die
\passthrough{\lstinline!tcp\_drop()!} Funktion und nutzt anschließend
\passthrough{\lstinline!printf!} Funktion, um die Informationen in den
Userspace zu loggen.

\begin{lstlisting}[language={C++}]
// tcpdrop.bt - SIMPLIFIED
kprobe:tcp_drop
{
    // GET SOCKET INFORMATION
    $sk = ((struct sock *) arg0);
    $inet_family = $sk->__sk_common.skc_family;
    //ADRESSES
    $daddr = ntop($sk->__sk_common.skc_daddr);
    $saddr = ntop($sk->__sk_common.skc_rcv_saddr);
    // PORTS
    $dport = $sk->__sk_common.skc_dport;
    $dport = $sk->__sk_common.skc_dport;
    //LOG INTO USERSPACE
    printf("%39s:%-6d %39s:%-6d %-10s\n", $saddr, $lport, $daddr, $dport, $statestr);
}
\end{lstlisting}

Um eine Lastspitze auf dem System zu erzeugen, wurde das
Netzwerkbenchmark-Tool \passthrough{\lstinline!ntttcp!}\cite{ntttcp}
verwendet. Mit diesem ist es möglich, UDP und \gls{tcp} Pakete mit
verschiedenen Paketgrößen zu generieren. Hierzu werden zwei Instanzen
benötigt, der Server und der Client, welche auf dem gleichen System aber
auch auf verschiedenen Systemen ausgeführt werden können.

\hypertarget{aufzeichnung-trace-log}{%
\section{Aufzeichnung Trace-Log}\label{aufzeichnung-trace-log}}

Um die Messung zu starten, wurde zuerst der
\passthrough{\lstinline!ntttcp!}-Server gestartet; dieser empfängt die
vom Sender gesendeten Pakete. Im zweiten Schritt wurde der
\passthrough{\lstinline!ntttcp!}-Client auf dem anderen System
gestartet. Hier wurde mittels \passthrough{\lstinline!-t!} Parameter die
Laufzeit auf unendlich gestellt, somit werden durchgehend Pakete an den
Server gesendet. Die Paketgröße wurde hier auf
\passthrough{\lstinline!4096Kbyte!} gestellt umso eine Fragmentierung
des \gls{tcp}-Paketes bei einer MTU von
\passthrough{\lstinline!1500byte!} zu erzwingen.

Im Anschluss wurde \passthrough{\lstinline!bpftrace!} gestartet, welches
die Events als Logdatei \passthrough{\lstinline!tcpdrop\_log!} in einem
lesbaren Textformat ausgeben soll.

\begin{lstlisting}[language=bash]
# START SERVER
$ ntttcp -r
NTTTCP for Linux 1.4.0
---------------------------------------------------------
21:27:58 INFO: 17 threads created


# RUN bpftrace RECORD
$ sudo bpftrace -o ~/tcpdrop_log -f text -v ~/tcpdrop.bt 
INFO: node count: 171
Program ID: 146
The verifier log: 
processed 374 insns (limit 1000000) max_states_per_insn 0
Attaching BEGIN
[...]

# START CLIENT # PACKET SIZE 16Byte
$ ntttcp -s10.11.12.1 -t -l 16
NTTTCP for Linux 1.4.0
---------------------------------------------------------
21:28:52 INFO: running test in continuous mode.
21:28:52 INFO: 64 threads created
21:28:52 INFO: 64 connections created in 5656 microseconds
21:28:52 INFO: Network activity progressing...
\end{lstlisting}

Nach einigen Sekunden wurde \passthrough{\lstinline!ntttcp!} und
\passthrough{\lstinline!bpftrace!} die Aufzeichnung manuell gestoppt.
Das aufgezeichnete Trace für das
\passthrough{\lstinline!tcp\_drop!}-Event befindet sich in der
\passthrough{\lstinline!tcpdrop\_log!} Datei.

\hypertarget{ausgabe}{%
\section{Ausgabe}\label{ausgabe}}

Die Ausgabe der Logdatei stellt Textbasiert nicht nur dar, ob ein
\gls{tcp}-Paket verloren wurde, sondern gibt auch zusätzliche
Informationen aus. Jeder Event-Trigger des
\passthrough{\lstinline!tcp\_drop()!} Events wird dabei mit der
Systemzeit, Prozess-ID und dem Programm eingeleitet unter welches das
Event ausgelöst hat. In diesem Fall wurde der Paketverlust durch ein
Empfangenes Paket der \passthrough{\lstinline!ntttcp!}-Anwendung
ausgelöst. Die Senderichtung des Pakets kann anhand der Quell- und
Empfangs-IP-Adresse ermittelt werden. Danach folgt der
Kernel-Stacktrace, in welchem der Funktionsaufruf-Verlauf bis zum
Auslösen des überwachten Events aufgeführt ist.

\begin{lstlisting}[language=bash]
$ cat ~/tcpdrop_log
[..]
# tcp_drop() TIME   PID  APPLICATION    SOURCE  DESTINATION
21:36:57 18157 ntttcp 10.11.12.1:5014 10.11.12.2:59012
        #CALLSTACK
        # LAST FUNCTION CALL
        tcp_drop+1
        tcp_v4_do_rcv+196
        __release_sock+120
        __tcp_close+444
        tcp_close+37
        inet_release+72
        __sock_release+66
        sock_close+21
        __fput+156
        ____fput+14
        task_work_run+112
        exit_to_user_mode_prepare+437
        syscall_exit_to_user_mode+39
        do_syscall_64+110
        entry_SYSCALL_64_after_hwframe+68
        # FIRST FUNCTION CALL
[...]
\end{lstlisting}

Somit ist aus den Logs zu entnehmen, dass unter den getesteten
Bedingungen auf dem System \gls{tcp}-Pakete verloren gingen, eine
tiefergehende Untersuchung des Kernel-Stacktrace kann hierzu genauere
Informationen bereitstellen. Das Beispiel zeigt auch, dass nicht nur das
Auslösen von Events protokolliert werden kann, sondern auch mittels
einfacher Skript-Befehle komplexe Debug-Informationen systematisch
gewonnen werden können.

\hypertarget{beispiel---identifikation-von-laufzeitproblemen}{%
\chapter{Beispiel - Identifikation von
Laufzeitproblemen}\label{beispiel---identifikation-von-laufzeitproblemen}}

In diesem Abschnitt soll an einem einfachen Beispiel gezeigt werden, wie
es mittels Tracing möglich ist, eine Analyse der Nachrichten auf einem
\gls{i2c}-Bus durchzuführen.

\hypertarget{ausgangsszenario}{%
\section{Ausgangsszenario}\label{ausgangsszenario}}

Als Ausgabgspunkt für dieses Beispiel, kommuniziert ein
Eingebettetes-System mit einem Sensor über den \gls{i2c}-Bus. Das
Programm welches mit dem Sensor kommuniziert ist eine Black-Box. Somit
steht kein Quellcode zur Verfügung. Hier soll das Protokoll analysiert
werden, um dieses in einer späteren Anwendung nachbauen zu können. Auch
gibt es hier keinen Zugriff auf die elektrische Ebene des Bus-Systems,
somit kann kein Logic-Analyzer verwendet werden.

\hypertarget{aufzeichnung-trace-log-1}{%
\section{Aufzeichnung Trace-Log}\label{aufzeichnung-trace-log-1}}

Zur Aufzeichnung des Trace-Logs wurde
\passthrough{\lstinline!trace-cmd!}\cite{trace-cmd} verwendet. Auf
dem Zielsystem wurde dabei nur die Aufzeichnung vorgenommen und die
Analyse der Logs erfolgte auf einem seperaten System. Für den Test wird
zuerst die \passthrough{\lstinline!Tracing!}-Funktionalität aktiviert
und alle \passthrough{\lstinline!sched!} und
\passthrough{\lstinline!gpio!}-Events aktiviert.

\begin{lstlisting}[language=bash]
# DANGER: RUN ALL NEXT COMMANDS AS ROOT
$ sudo su
# ACTIVATE DEBUG FS
$ mount -t debugfs none /sys/kernel/debug
# USE NOP TRACER
$ echo nop > current_tracer
# CLEAR RECENT EVENT LOG
$ echo > /sys/kernel/debug/tracing/trace
# ENABLE ALL I2C-BUS EVENTS
echo 1 > /sys/kernel/debug/tracing/events/i2c/enable
# ENABLE TRACING
$ echo 1 > /sys/kernel/debug/tracing/tracing_on
\end{lstlisting}

Nachdem das Tracing aktiviert wurde, wurde das Trace-Log manuell
analysiert.

\begin{lstlisting}[language=bash]
$ cat /sys/kernel/debug/tracing/trace
# tracer: nop
#
# entries-in-buffer/entries-written: 96/96 
#           _-----=> irqs-off
#          / _----=> need-resched
#         | / _---=> hardirq/softirq
#         || / _--=> preempt-depth
#         ||| / _-=> migrate-disable
#         |||| /     delay
#TASK-PID |||||  TIMESTAMP  FUNCTION
# |   |   |||||     |         |
7127  987.236090: i2c_write: i2c-1 #0 a=020 f=0000 l=1 [01]
7127  987.236656: i2c_result: i2c-1 n=1 ret=1
7127  987.737266: i2c_write: i2c-1 #0 a=020 f=0000 l=1 [02]
7127  987.737827: i2c_result: i2c-1 n=1 ret=1
7127  988.238418: i2c_write: i2c-1 #0 a=020 f=0000 l=1 [04]
7127  988.238977: i2c_result: i2c-1 n=1 ret=1
7127  988.739545: i2c_write: i2c-1 #0 a=020 f=0000 l=1 [08]
7127  988.740132: i2c_result: i2c-1 n=1 ret=1
[...]
\end{lstlisting}

\hypertarget{auswertung}{%
\section{Auswertung}\label{auswertung}}

Das Tracelog zeigt nun einige Events vom Typ
\passthrough{\lstinline!i2c\_write!}. Diese werden für die Analyse
benötigt, da hier die vom System über den \gls{i2c}-Bus gesendeten
Nachrichten stehen. Zu sehen ist, dass über den
\passthrough{\lstinline!i2c-1!} Bus des Systems gesendet wird. Die
Zieladresse ist dabei \passthrough{\lstinline!0x20!} und wird im Log mit
dem Präfix \passthrough{\lstinline!a=!} angegeben. Die Länge der
gesendeten Bytes ist mit \passthrough{\lstinline!l=1!} angegeben. Somit
wurde nur ein Byte an den Slave gesendet. Die eigentlichen Daten sind in
Hex-Array-Schreibweise angegeben. Hier wurde nacheinander
\passthrough{\lstinline![01]!}, \passthrough{\lstinline![02]!},
\passthrough{\lstinline![04]!}, \passthrough{\lstinline![08]!}
gesendet.\cite{bussnooping}

Die korrektheit der Adresse kann zusätzlich mittels des
\passthrough{\lstinline!i2cdetect!}-Befehls überprüft werden. Die
Ausgabe zeigt, dass am System nur ein \gls{i2c}-Slave mit der Adresse
\passthrough{\lstinline!0x20!} angeschlossen ist, welches somit mit der
Ausgabe im Trace-Log übereinstimmt.

\begin{lstlisting}[language=bash]
$ i2cdetect -y 1
     0  1  2  3  4  5  6  7  8  9  a  b  c  d  e  f
00:                         -- -- -- -- -- -- -- -- 
10: -- -- -- -- -- -- -- -- -- -- -- -- -- -- -- -- 
20: 20 -- -- -- -- -- -- -- -- -- -- -- -- -- -- --
\end{lstlisting}

Anhand der I2C-Adresse und der einfachheit der gesendeten Daten, kann
auf einen \passthrough{\lstinline!PCF8574!} Port-Expander-IC geschlossen
werden. Dieser nimmt jeweils an Adresse \passthrough{\lstinline!0x20!}
ein Byte entgegen und schaltet somit die jeweiligen Ausgangspins. Der
Quellcode bestätigt diese Erkenntnisse ebenfalls. Die in Python
geschriebene ``Black-Box'' verwendet das
\passthrough{\lstinline!smbus!}-Modul um auf den \gls{i2c}-Bus-1
zuzugreifen und sendet in einer Dauerschleife jeweils ein Byte.

\begin{lstlisting}[language=Python]
#!/bin/env python3
import smbus
import time
# USE I2C-BUS 1
bus = smbus.SMBus(1)
 
while True:
    bus.write_byte(0x20, 0x01)
    time.sleep(0.5)
    bus.write_byte(0x20, 0x02)
    time.sleep(0.5)
    bus.write_byte(0x20, 0x04)
    time.sleep(0.5)
    bus.write_byte(0x20, 0x08)
    time.sleep(0.5)
\end{lstlisting}

Somit konnte das Protokoll des \gls{i2c}-Slave mittels Tracing ohne
Kenntnis der Software analysiert und reverse engineered werden.