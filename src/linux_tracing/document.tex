\hypertarget{einleitung}{%
\chapter{Einleitung}\label{einleitung}}

Tracing ist die spezielle Verwendung der Protokollierung zur
Aufzeichnung von Informationen über den Ausführungsablauf eines
Programms. Oft werden mit eigenständig hinzugefügte Print-Messages der
Code debuggt. Somit verfolgt man die Anweisungen mit einem eigenem
tracing-System. Linux bringt einige eigenständige Tools mit, mit denen
es möglich ist Vorgänge innerhalb von einem Embedded-System
nachvollziehen und analysieren zu können. Die Linux-Tracing
Funktionalität und die bestehenden Tools, welche im Linux-Kernel
integriert sind, helfen so dabei bei der Identifikation von Laufzeiten,
Nebenläufigkeiten und der Untersuchung von Latenzproblemen,

\hypertarget{tracing}{%
\section{Tracing}\label{tracing}}

Nachfolgend wird erleutert und an einem Beispiel demonstriert, wie das
Linux-Tracing bei der Identifikation von Laufzeitproblemen eingesetzt
werden kann.

\hypertarget{ursprung}{%
\section{Ursprung}\label{ursprung}}

\hypertarget{grundlagen}{%
\chapter{Grundlagen}\label{grundlagen}}

\hypertarget{events}{%
\section{Events}\label{events}}

\hypertarget{kprobes}{%
\section{Kprobes}\label{kprobes}}

Kprobes können dazu verwendet werden, Laufzeit und Performance-Daten des
Kernels zu sammeln. Der Vorteil and diesen ist, dass diese Daten ohne
Unterbrechnung der Ausführung auf CPU-Instruktions-Ebene aggregiert
werden können, anders wie bei dem Debuggen eines Programms mittels
Breakpoints.

\hypertarget{cpu-traps}{%
\subsection{CPU-Traps}\label{cpu-traps}}

\hypertarget{tracing-auf-mikrokontrollern}{%
\chapter{Tracing auf
Mikrokontrollern}\label{tracing-auf-mikrokontrollern}}

\hypertarget{tools}{%
\chapter{Tools}\label{tools}}

Allgemein sind keine speziellen Programme notwending um die
Laufzeiteigenschaften eines Programms aufzuzeichnen. Der Linux-Kernel
bringt bereits alle nötigen Funktionalitäten mit. Jedoch gibt es Tools
die eine visuelle Darstellung der aufgezeichneten Events ermöglichen.

\hypertarget{trace-log-aufzeichnung}{%
\section{Trace-Log Aufzeichnung}\label{trace-log-aufzeichnung}}

\hypertarget{ftrace}{%
\subsection{ftrace}\label{ftrace}}

\hypertarget{visualisierung}{%
\section{Visualisierung}\label{visualisierung}}

\hypertarget{kernelshark}{%
\subsection{Kernelshark}\label{kernelshark}}

\begin{figure}
\centering
\includegraphics{images/kernelshark.png}
\caption{Kernelshark \label{kernelshark}}
\end{figure}

\hypertarget{interpretation-des-kernel-trace-ergebnisses}{%
\chapter{Interpretation des Kernel-Trace
Ergebnisses}\label{interpretation-des-kernel-trace-ergebnisses}}

\hypertarget{beispiel-der-identifikation-von-laufzeitproblemen}{%
\chapter{Beispiel der Identifikation von
Laufzeitproblemen}\label{beispiel-der-identifikation-von-laufzeitproblemen}}

\hypertarget{ausgangsszenario}{%
\section{Ausgangsszenario}\label{ausgangsszenario}}

Als Ausgangspunkt dieses Beispiels, soll das Laufzeitverhalten eines
Programms auf einem Linux-System analysiert werden. Die zugrunde
liegende Software wurde bisher nur auf einem Linux-Realtime Kernel
verwendet, jedoch erfordert die Implementation neuer Features eine
neuere Kernel-Version, welche noch nicht als RT-Version auf dem System
zur Verfügung steht. Somit soll ermittelt werden, ob die unmodifizierte
Software eins zu eins auf dem neuen System lauffähig ist und die
Laufzeitandorderungen erfüllt.

\hypertarget{aufzeichnung-mittels-ftrace}{%
\section{Aufzeichnung mittels
ftrace}\label{aufzeichnung-mittels-ftrace}}

\hypertarget{visualisierung-und-beurteilung-des-trace-logs-mittels-kernelshark}{%
\section{Visualisierung und Beurteilung des Trace-Logs mittels
kernelshark}\label{visualisierung-und-beurteilung-des-trace-logs-mittels-kernelshark}}